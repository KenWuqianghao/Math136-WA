\documentclass[11pt]{article}
\usepackage{amsmath,amssymb,amsthm,enumerate,nicefrac,fancyhdr,hyperref,graphicx,adjustbox}
\hypersetup{colorlinks=true,urlcolor=blue,citecolor=blue,linkcolor=blue}
\usepackage[left=2.6cm, right=2.6cm, top=1.5cm, includehead, includefoot]{geometry}
\usepackage[dvipsnames]{xcolor}
\usepackage[d]{esvect}

%% commands
%% useful macros [add to them as needed]
% sets
\newcommand{\C}{{\mathbb{C}}} 
\newcommand{\N}{{\mathbb{N}}}
\newcommand{\Q}{{\mathbb{Q}}}
\newcommand{\R}{{\mathbb{R}}}
\newcommand{\Z}{{\mathbb{Z}}}
\newcommand{\F}{{\mathbb{F}}}

% bases
\newcommand{\mA}{\mathcal{A}}
\newcommand{\mB}{\mathcal{B}}
\newcommand{\mC}{\mathcal{C}}
\newcommand{\mD}{\mathcal{D}}
\newcommand{\mE}{\mathcal{E}}
\newcommand{\mL}{\mathcal{L}}
\newcommand{\mM}{\mathcal{M}}
\newcommand{\mO}{\mathcal{O}}
\newcommand{\mP}{\mathcal{P}}
\newcommand{\mS}{\mathcal{S}}
\newcommand{\mT}{\mathcal{T}}

% linear algebra
\newcommand{\diag}{\operatorname{diag}}
\newcommand{\adj}{\operatorname{adj}}
\newcommand{\rank}{\operatorname{rank}}
\newcommand{\spn}{\operatorname{Span}}
\newcommand{\proj}{\operatorname{proj}}
\newcommand{\prp}{\operatorname{perp}}
\newcommand{\refl}{\operatorname{refl}}
\newcommand{\tr}{\operatorname{tr}}
\newcommand{\nul}{\operatorname{Null}}
\newcommand{\nully}{\operatorname{nullity}}
\newcommand{\range}{\operatorname{Range}}
\renewcommand{\ker}{\operatorname{Ker}}
\newcommand{\col}{\operatorname{Col}}
\newcommand{\row}{\operatorname{Row}}
\newcommand{\cof}{\operatorname{cof}}
\newcommand{\Num}{\operatorname{Num}}
\newcommand{\Id}{\operatorname{Id}}
\newcommand{\ipb}{\langle \thinspace, \rangle}
\newcommand{\ip}[2]{\left\langle #1, #2\right\rangle} % inner products
\newcommand{\M}[2]{M_{#1\times #2}(\F)}
\newcommand{\RREF}{\operatorname{RREF}}
\newcommand{\cv}[1]{\begin{bmatrix} #1 \end{bmatrix}}
\newenvironment{amatrix}[1]{\left[\begin{array}{@{}*{\numexpr#1-1}{c}|c@{}}}{\end{array}\right]}
\newcommand{\am}[2]{\begin{amatrix}{#1} #2 \end{amatrix}}

% vectors
\newcommand{\vzero}{\vv{0}}
\newcommand{\va}{\vv{a}}
\newcommand{\vb}{\vv{b}}
\newcommand{\vc}{\vv{c}}
\newcommand{\vd}{\vv{d}}
\newcommand{\ve}{\vv{e}}
\newcommand{\vf}{\vv{f}}
\newcommand{\vg}{\vv{g}}
\newcommand{\vh}{\vv{h}}
\newcommand{\vl}{\vv{\ell}}
\newcommand{\vm}{\vv{m}}
\newcommand{\vn}{\vv{n}}
\newcommand{\vp}{\vv{p}}
\newcommand{\vq}{\vv{q}}
\newcommand{\vr}{\vv{r}}
\newcommand{\vs}{\vv{s}}
\newcommand{\vt}{\vv{t}}
\newcommand{\vu}{\vv{u}}
\newcommand{\vvv}{{\vv{v}}}
\newcommand{\vw}{\vv{w}}
\newcommand{\vx}{\vv{x}}
\newcommand{\vy}{\vv{y}}
\newcommand{\vz}{\vv{z}}

% display
\newcommand{\ds}{\displaystyle}
\newcommand{\qand}{\quad\text{and}}
\newcommand{\qandq}{\quad\text{and}\quad}
\newcommand{\hint}{\textbf{Hint: }}

% misc
\newcommand{\area}{\operatorname{area}}
\newcommand{\vol}{\operatorname{vol}}
\newcommand{\red}[1]{{\color{red} #1}}
\newcommand{\rc}{\red{\checkmark}}

%% header
\pagestyle{fancy}
\fancyhead[L]{\bf\large MATH136: Linear Algebra 1 \\ Written Assignment 3 Solutions}
\fancyhead[R]{\bf\large Winter 2023 \\}
%\fancyfoot[C]{Page \thepage\ of 2}
\setlength{\headheight}{35pt}

\begin{document}
	\begin{enumerate}[{\bf Q1.}]	
		%Q1
		\item
		\begin{enumerate}
			%Q1(a)
			
			\item
			\begin{equation}
				\begin{aligned}
				A^2 & =[T]_{\varepsilon}^2 \\
				& =[T \circ T]_{\varepsilon} \vec{x}
				\end{aligned}
				\end{equation}
			\begin{equation}
				=2 \operatorname{proj}_{\vec{v}}\left(2 \operatorname{proj}_{\vec{v}} \vec{x}+2 \operatorname{proj}_{\vec{w}} \vec{x}-\vec{x}\right)+2 \operatorname{proj}_{\vec{w}}\left(2 \operatorname{proj}_{\vec{v}} \vec{x}+2 \operatorname{proj}_{\vec{w}} \vec{x}-\vec{x}\right)-\vec{x}
				\end{equation}
				\begin{equation}
					=2 \operatorname{proj}_{\vec{v}}\left(2 \operatorname{proj}_{\vec{v}} \vec{x}\right)+2 \operatorname{proj}_{\vec{v}}\left(2 \operatorname{proj}_{\vec{w}} \vec{x}\right)-2 \operatorname{proj}_{\vec{v}} \vec{x} +2 \operatorname{proj}_{\vec{w}}\left(2 \operatorname{proj}_{\vec{v}} \vec{x}\right)+2 \operatorname{prog}_{\vec{w}}\left(2 \operatorname{proj}_{\vec{w}} \vec{x}\right)-2 \operatorname{proj}_{\vec{w}} \vec{x}-\vec{x}
				\end{equation}
			\begin{equation}
				=4 \operatorname{proj}_{\vec{v}} \vec{x}+\overrightarrow{0}-2 \operatorname{proj}_{\vec{v}} \vec{x}+\overrightarrow{0} +4 \operatorname{proj}_{\vec{w}} \vec{x}-2 \operatorname{proj}_{\vec{w}} \vec{x}-\vec{x}
				\end{equation}
			\begin{equation}
				=2\left(\operatorname{proj}_{\vec{v}} \vec{x}+\operatorname{proj}_{\vec{w}} \vec{x}\right)-\vec{x}
				\end{equation}
			\begin{equation}
				=2 \vec{x}-\vec{x}=\bar{x}
				\end{equation}
			%Q1(b)
			\item 
			From part a we know there exist $A^{-1}=A$ s.t.
			$$
			A^{-1} A=[T \circ T]_{\varepsilon}=A A^{-1}=I_3
			$$
			$\Rightarrow A$ is invertible
			Therefore, by invertibility criteria - second version,
			$T_A$ is onto and one to one.

			%Q1(c)
			\item 
			\begin{equation}
				\begin{aligned}
				T\left(\left[\begin{array}{l}
				a \\
				b \\
				c
				\end{array}\right]\right) & =2\left[\begin{array}{l}
				a \\
				0 \\
				0
				\end{array}\right]+2\left[\begin{array}{l}
				0 \\
				b \\
				0
				\end{array}\right]-\left[\begin{array}{l}
				a \\
				b \\
				c
				\end{array}\right] \\
				& =\left[\begin{array}{c}
				2 a \\
				2 b \\
				0
				\end{array}\right]-\left[\begin{array}{l}
				a \\
				b \\
				c
				\end{array}\right] \\
				& =\left[\begin{array}{c}
				a \\
				b \\
				-c
				\end{array}\right]
				\end{aligned}
				\end{equation}

			%Q1(d)
			\item 
			Reflects the vector across one of the $x, y$ or $z$ axis.

		\end{enumerate}
		
		% New page for Crowdmark
		\newpage
		
		%Q2
		\item 
		\begin{enumerate}
			%Q2(a)
			\item 
			\begin{equation}
				A_1=\left[\begin{array}{cc}
				0 & 3 \\
				-3 & 0
				\end{array}\right] \quad A_2=\left[\begin{array}{cccc}
				0 & 3 & 3 & 3 \\
				-3 & 0 & 3 & 3 \\
				-3 & -3 & 0 & 3 \\
				-3 & -3 & -3 & 0
				\end{array}\right]
				\end{equation}

			%Q2(b)
			\item 
			$b_1 \operatorname{det}\left(A_n\right)=3^{2 n}$
			Let $A_n^{-1}=-\frac{1}{3} A_n, A_n^{-1}$ is the inverse of $A_n$ meaning $A_n$ is invertible \newline
			$\Rightarrow A_n$ is row equivalent to $I_{2 n}$ \newline
			$I_{2 n}$ can undergo $2 n$ number of Row scale to become the following
			$$
			\left[\begin{array}{cccccc}
			3 & 0 & 0 & 0 & \cdots & 0 \\
			0 & 3 & 0 & 0 & \cdots & 0 \\
			0 & 0 & 3 & 0 & \cdots & 0 \\
			\cdots & \cdots & \cdots & \cdots & \cdots \\
			0 & 0 & 0 & 0 & \cdots & 3
			\end{array}\right]
			$$
			It is apparent after some row addition, $n$ row swaps and $n$ row scale with -1 following the Gauss-Jordan elimination method. \newline
			By Theorem 6.2.1, Effect of ERO on
			Determinent, we can conclude $\operatorname{det}\left(A_n\right)=1 \cdot 3^{2 n} \cdot(-1)^{2 n}=3^{2 n}$
			%Q2(c)
			\item 
			By Theorem 6.3.1, since $\operatorname{det}\left(A_n\right) \neq 0$, \newline
			$\Rightarrow A_n$ is always invertible.

		\end{enumerate}
		
		% New page for Crowdmark
		\newpage
		
		%Q3 
		\item
		\begin{enumerate}
			%Q3(a)
			\item 
			\begin{equation}
				\begin{aligned}
				& \text { i. Let } z=a+b_i \quad w=c + d i \\
				& f(2+w)=f(a+b i+c+d i)=f((a+c)+(b+d) i) \\
				& =\left[\begin{array}{cc}
				a+c & -(b+d) \\
				b+d & a+c
				\end{array}\right] \\
				& =\left[\begin{array}{cc}
				a & -b \\
				b & a
				\end{array}\right]+\left[\begin{array}{cc}
				c & -d \\
				d & c
				\end{array}\right] \\
				& =f(z)+f(w) \\
				& \text { ii. Let } 2=a+b i \\
				& f(t z)=f(t a+t b i) \\
				& =\left[\begin{array}{cc}
				t a & -t b \\
				t b & t a
				\end{array}\right]=t\left[\begin{array}{cc}
				a & -b \\
				b & a
				\end{array}\right] \\
				& =t f(z) \\
				&
				\end{aligned}
				\end{equation}

			%Q3(b)
			\item 
			\begin{equation}
				\text { b. Let } \begin{aligned}
				z & =a+b i \quad w=c+d i \\
				f(z w) & =f(c a+b i)(c+d i)) \\
				& =f(a c+a d i+b c i-b d) \\
				& =f((a c-b d)+(a d+b c) i) \\
				& =\left[\begin{array}{cc}
				a(-b d & -a d-c b \\
				a d+c b & a c-b d
				\end{array}\right] \\
				& =\left[\begin{array}{cc}
				a & -b \\
				b & a
				\end{array}\right]\left[\begin{array}{cc}
				c & -d \\
				d & c
				\end{array}\right] \\
				& =f(z) \cdot f(w)
				\end{aligned}
				\end{equation}

			%Q3(c)
			\item 
			$$
			\begin{aligned}
			& \text { C. Let } z=a+b i \quad w=c + d i \quad \text { and } f(z)=f(w) \\
			& f\left(a+b_i\right)=f\left(c+d_i\right) \\
			& {\left[\begin{array}{cc}
			a & -b \\
			b & a
			\end{array}\right]=\left[\begin{array}{cc}
			c & -d \\
			d & c
			\end{array}\right]} \\
			& \Rightarrow a=c, b=d \\
			& \Rightarrow a+b i=c+d i \\
			& z=w \\
			&
			\end{aligned}
			$$
			Hence, $f$ is one to one

		\end{enumerate}
		
		% New page for Crowdmark
		\newpage
		
		%Q4
		\item
		\begin{enumerate}
			%Q4(a)
			\item 
			Proof by contradiction \newline
			Assume all entries of $A$ are real and $A^2=-I_n$ \newline 
			$\operatorname{det}\left(A^2\right)=\operatorname{det}\left(-I_n\right)$
			$$
			\begin{aligned}
			& =-1(n \text { is } \operatorname{odd}) \\
			\operatorname{det}\left(A^2\right)= & \operatorname{det}(A) \operatorname{det}(A)=(\operatorname{det}(A))^2=-1 \\
			\operatorname{det} A & =i
			\end{aligned}
			$$
			However, this contradict gar assumption that all entries are veal, otherwise we cant produce a determinant of $i$, hence the contradiction. \newline
			QED

			%Q4(b)
			\item 
			\begin{equation}
				\begin{aligned}
				& \text { Let } A=\left[\begin{array}{cc}
				0 & -1 \\
				1 & 0
				\end{array}\right] \\
				& A^2=\left[\begin{array}{cc}
				0 & -1 \\
				1 & 0
				\end{array}\right]\left[\begin{array}{cc}
				0 & -1 \\
				1 & 0
				\end{array}\right] \\
				& =-I_2 \\
				&
				\end{aligned}
				\end{equation}
				
			%Q4(b) [bonus]
			\item 
			Let $A \in M_{h \times n}(\mathbb{R})$ whose $i, j$ th entries ave given by
			$$
			a_{i, j}= \begin{cases}-1 & \text { if } i<j \text { and } j=n+1-i \\ 1 & \text { if } i>j \text { and } i=n+1-j \\ 0 & \text { otherwise }\end{cases}
			$$
			\begin{equation}
				\begin{aligned}
				A^2=A A & =\left[\begin{array}{ccccc}
				0 & 0 & \cdots & 0 & -1 \\
				0 & 0 & \cdots & -1 & 0 \\
				\cdots & \cdots & \cdots & \cdots & 0 \\
				0 & 1 & \cdots & 0 \\
				1 & 0 & \cdots & 0 & 0
				\end{array}\right]\left[\begin{array}{ccccc}
				0 & 0 & \cdots & 0 & -1 \\
				0 & 0 & \cdots & -1 & 0 \\
				\cdots & \cdots & \cdots & \cdots & \cdots \\
				0 & 1 & \cdots & 0 & 0 \\
				1 & 0 & \cdots & 0 & 0
				\end{array}\right] \\
				& =\left[\begin{array}{ccccc}
				-1 & 0 & \cdots & 0 & 0 \\
				0 & -1 & \cdots & 0 & 0 \\
				\cdots & \cdots & \cdots & \cdots & \cdots \\
				0 & 0 & \cdots & -1 & 0 \\
				0 & 0 & \cdots & 0 & -1
				\end{array}\right]=-I_n
				\end{aligned}
				\end{equation}

		\end{enumerate}
	\end{enumerate}
\end{document}