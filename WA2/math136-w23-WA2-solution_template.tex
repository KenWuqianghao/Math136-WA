\documentclass[11pt]{article}
\usepackage{amsmath,amssymb,amsthm,enumerate,nicefrac,fancyhdr,hyperref,graphicx,adjustbox}
\hypersetup{colorlinks=true,urlcolor=blue,citecolor=blue,linkcolor=blue}
\usepackage[left=2.6cm, right=2.6cm, top=1.5cm, includehead, includefoot]{geometry}
\usepackage[dvipsnames]{xcolor}
\usepackage[d]{esvect}

%% commands
%% useful macros [add to them as needed]
% sets
\newcommand{\C}{{\mathbb{C}}} 
\newcommand{\N}{{\mathbb{N}}}
\newcommand{\Q}{{\mathbb{Q}}}
\newcommand{\R}{{\mathbb{R}}}
\newcommand{\Z}{{\mathbb{Z}}}
\newcommand{\F}{{\mathbb{F}}}

% bases
\newcommand{\mA}{\mathcal{A}}
\newcommand{\mB}{\mathcal{B}}
\newcommand{\mC}{\mathcal{C}}
\newcommand{\mD}{\mathcal{D}}
\newcommand{\mE}{\mathcal{E}}
\newcommand{\mL}{\mathcal{L}}
\newcommand{\mM}{\mathcal{M}}
\newcommand{\mO}{\mathcal{O}}
\newcommand{\mP}{\mathcal{P}}
\newcommand{\mS}{\mathcal{S}}
\newcommand{\mT}{\mathcal{T}}

% linear algebra
\newcommand{\diag}{\operatorname{diag}}
\newcommand{\adj}{\operatorname{adj}}
\newcommand{\rank}{\operatorname{rank}}
\newcommand{\spn}{\operatorname{Span}}
\newcommand{\proj}{\operatorname{proj}}
\newcommand{\prp}{\operatorname{perp}}
\newcommand{\refl}{\operatorname{refl}}
\newcommand{\tr}{\operatorname{tr}}
\newcommand{\nul}{\operatorname{Null}}
\newcommand{\nully}{\operatorname{nullity}}
\newcommand{\range}{\operatorname{Range}}
\renewcommand{\ker}{\operatorname{Ker}}
\newcommand{\col}{\operatorname{Col}}
\newcommand{\row}{\operatorname{Row}}
\newcommand{\cof}{\operatorname{cof}}
\newcommand{\Num}{\operatorname{Num}}
\newcommand{\Id}{\operatorname{Id}}
\newcommand{\ipb}{\langle \thinspace, \rangle}
\newcommand{\ip}[2]{\left\langle #1, #2\right\rangle} % inner products
\newcommand{\M}[2]{M_{#1\times #2}(\F)}
\newcommand{\RREF}{\operatorname{RREF}}
\newcommand{\cv}[1]{\begin{bmatrix} #1 \end{bmatrix}}
\newenvironment{amatrix}[1]{\left[\begin{array}{@{}*{\numexpr#1-1}{c}|c@{}}}{\end{array}\right]}
\newcommand{\am}[2]{\begin{amatrix}{#1} #2 \end{amatrix}}

% vectors
\newcommand{\vzero}{\vv{0}}
\newcommand{\va}{\vv{a}}
\newcommand{\vb}{\vv{b}}
\newcommand{\vc}{\vv{c}}
\newcommand{\vd}{\vv{d}}
\newcommand{\ve}{\vv{e}}
\newcommand{\vf}{\vv{f}}
\newcommand{\vg}{\vv{g}}
\newcommand{\vh}{\vv{h}}
\newcommand{\vl}{\vv{\ell}}
\newcommand{\vm}{\vv{m}}
\newcommand{\vn}{\vv{n}}
\newcommand{\vp}{\vv{p}}
\newcommand{\vq}{\vv{q}}
\newcommand{\vr}{\vv{r}}
\newcommand{\vs}{\vv{s}}
\newcommand{\vt}{\vv{t}}
\newcommand{\vu}{\vv{u}}
\newcommand{\vvv}{{\vv{v}}}
\newcommand{\vw}{\vv{w}}
\newcommand{\vx}{\vv{x}}
\newcommand{\vy}{\vv{y}}
\newcommand{\vz}{\vv{z}}

% display
\newcommand{\ds}{\displaystyle}
\newcommand{\qand}{\quad\text{and}}
\newcommand{\qandq}{\quad\text{and}\quad}
\newcommand{\hint}{\textbf{Hint: }}

% misc
\newcommand{\area}{\operatorname{area}}
\newcommand{\vol}{\operatorname{vol}}
\newcommand{\red}[1]{{\color{red} #1}}
\newcommand{\rc}{\red{\checkmark}}

%% header
\pagestyle{fancy}
\fancyhead[L]{\bf\large MATH136: Linear Algebra 1 \\ Written Assignment 2 Solutions}
\fancyhead[R]{\bf\large Winter 2023 \\}
%\fancyfoot[C]{Page \thepage\ of 2}
\setlength{\headheight}{35pt}

\begin{document}
	\begin{enumerate}[{\bf Q1.}]
		
		%Q1
		\item
		Proof by Induction \newline
			Base Case
			$$
			\begin{aligned}
			& n=2 \\
			& A^2=A A=\left[\begin{array}{lll}
			a & 1 & 0 \\
			0 & a & 1 \\
			0 & 0 & a
			\end{array}\right]\left[\begin{array}{lll}
			a & 1 & 0 \\
			0 & a & 1 \\
			0 & 0 & a
			\end{array}\right] \\
			& =\left[\begin{array}{ccc}
			a^2 & 2 a & 1 \\
			0 & a^2 & 2 a \\
			0 & 0 & a^2
			\end{array}\right] \\
			& =\left[\begin{array}{ccc}
			a^2 & 2 a^{2-1} & \frac{2(2-1)}{2} a^{2-2} \\
			0 & a^2 & 2 a^{2-1} \\
			0 & 0 & a^2
			\end{array}\right] \\
			&
			\end{aligned}
			$$
			Base case holds. \newline

			$$
			\begin{aligned}
			A^{k+1} & =A^k A \\
			& =\left[\begin{array}{ccc}
			a^k & k a^{k-1} & \frac{k(k-1)}{2} a^{k-2} \\
			0 & a^k & k a^{k-1} \\
			0 & 0 & a^k
			\end{array}\right] \cdot\left[\begin{array}{ccc}
			a & 1 & 0 \\
			0 & a & 1 \\
			0 & 0 & a
			\end{array}\right] \\
			& =\left[\begin{array}{ccc}
			a^{k+1} & a^k+k a^k & k a^{k-1}+\frac{k(k-1)}{2} a^{k-1} \\
			0 & a^{k+1} & a^k+k a^k \\
			0 & 0 & a^{k+1}
			\end{array}\right] \\
			& =\left[\begin{array}{ccc}
			a^{k+1} & (k+1) a^k & \frac{(k+1) k}{2} a^{k-1} \\
			0 & a^{k+1} & (k+1) a^k \\
			0 & 0 & a^{k+1}
			\end{array}\right]
			\end{aligned}
			$$
			Hence, by principle of mathematical induction, the statement is true. \newline
			QED
		% New page for Crowdmark
		\newpage
		
		%Q2
		\item 
		\begin{enumerate}
			%Q2(a)
			\item 
			Assume $L_1: \mathbb{F}^m \rightarrow \mathbb{F}^n$ and $L_2: \mathbb{F}^{n} \rightarrow \mathbb{F}^{\text {m}}$ are inverses of $T$ \newline
			$T\left(L_1(\vec{v})\right)=\vec{v}$, for all $\vec{v} \in \mathbb{F}^m$ (Definition of inverse) \newline
			$L_2\left(T\left(L_1(\vec{v}))\right)=L_2(\vec{v})\right.$ (Apply $L_2$ to both sides of the equation) \newline
			$L_1 \vec{v}=L_2 \vec{v} \quad$ (Definition of inverse) \newline
			QED \newline

			%Q2(b)
			\item 
			Assume $T: \mathbb{F}^n \rightarrow \mathbb{F}^{m}$ is an invertible linear transformation and
			$L: \mathbb{F}^m \rightarrow \mathbb{F}^n$ is its inverse
			$$
			\begin{aligned}
			& L(T(\vec{x}+\vec{y})) \\
			= & L(T(\vec{x})+T(\vec{y})) \text { (Definition of Linear Transformation) } \\
			= & \vec{x}+\vec{y}(\text { Definition of Inverse) } \\
			= & L(T(\vec{x}))+L(T(\vec{y})) \text { (Definition of Linear Transformation) }
			\end{aligned}
			$$
			Since we know $T$ is linear transformation, it is also onto.
			$$
			\begin{aligned}
			& \Rightarrow\{T(\vec{x})+T(\vec{y})\}=\mathbb{F}^m \\
			& L(T(c \vec{x})) \\
			&= L(C T(\vec{x})) \text { (Definition of Linear Transformation) } \\
			&=c\vec{x} \text { (Definition of Inverse) } \\
			&= c L(T(\vec{x})) \text { (Definition of Inverse) } \\
			& \text { Similarly, since } T \text { is onto } \\
			& \Rightarrow\{T(c \vec{x})\}=\mathbb{F}^m
			\end{aligned}
			$$
			since $L$ satisfies the clefinition of a linear transformation, $L$ is a linear transformation.
			QED
			%Q2(c)
			\item
			Assume $T_A: \mathbb{F}^n \rightarrow \mathbb{F}^m$ is a linear trans formation determined by $A \in M_{u \times n}(\mathbb{F})$
			\newline $\Rightarrow$ \newline
			Assume $T_A$ is invertible,
			$$
			T_A=A \vec{x}, \vec{x} \in \mathbb{F}^n
			$$
			Let $L$ also be determined by a matrix $B$
			$$
			\begin{aligned}
			& L_B=B \vec{x}, \vec{x} \in \mathbb{F}^m \\
			& L_B\left(T_A(\vec{x})\right)=\vec{x} \text { (Definition of Inverse) } \\
			& L_B(A \vec{x})=\vec{x} \\
			& B A \vec{x}=\vec{x} \\
			& B A=I_n
			\end{aligned}
			$$
			$\Rightarrow A$ is invertible when $L$ is determined by $B$, inverse of $A$
			\newline $\Leftarrow$ \newline 
			Assume $A$ is invertable
			$$
			\begin{aligned}
			A A^{-1} & =I_n \text { (Definition of Invert table) } \\
			A A^{-1} \vec{x} & =\vec{x} \\
			T_A\left(A^{-1} \vec{x}\right) & =\vec{x}
			\end{aligned}
			$$
			Also ,
			$$
			\begin{aligned}
			& A^{-1} A \vec{x}=\vec{x} \\
			& A^{-1} T_A(\vec{x})=\vec{x}
			\end{aligned}
			$$
			Hence, we can create a matrix function $L: \mathbb{F}^n \rightarrow \mathbb{F}^m$ determined by matrix $A^{-1}$ where when $A^{-1}$ subbed into above equations
			$$
			\begin{aligned}
			& T_A\left(L_{A^{-1}}(\vec{x})\right)=\vec{x} \\
			& L_{A^{-1}}\left(T_A(\vec{x})\right)=\vec{x}
			\end{aligned}
			$$
			which satisfies the definition of invertability for $T_A$. \newline
			QED \newline
		\end{enumerate}
		
		% New page for Crowdmark
		\newpage
		
		%Q3 
		\item
		Direct Proof \newline
		Let $\vec{u}$ be an aubitvary vector $\in \mathbb{F}^n$,
		since $\mathbb{F}^h=\operatorname{span}\left\{\vec{v}_1 \ldots v_k\right\}$
		$$
		\begin{aligned}
		\vec{u}=a \vec{v}_1+\cdots & +k \vec{v}_k \\
		T(\vec{u}) & =T\left(a \vec{v}_1+\cdots+k \vec{v}_k\right) \\
		& =a T\left(\vec{v}_1\right)+\cdots+k T\left(\vec{v}_k\right) \quad \text { (Definition of Linear Transformation) } \\
		& =a S\left(\overrightarrow{v_1}\right)+\cdots+k s\left(\vec{v}_k\right) \text { (Sub with } T\left(\vec{v}_j\right)=s\left(\vec{v}_i\right) \\
		& =s\left(a \vec{v}_1\right)+\cdots+s\left(k \vec{v}_k\right) \quad \text { (Definition of Linear Transformation) } \\
		& =s\left(a \vec{v}_1+\cdots+k \vec{v}_k\right) \quad \text { (Definition of Linear Transformation) } \\
		& =s(\vec{u})
		\end{aligned}
		$$
		since $\vec{u}$ is arbitrary, \newline
		$T(\vec{v})=S(\vec{v})$ for all $\vec{v} \in \mathbb{F}^h$
		$$
		\begin{aligned}
		& \text { QED } \\
		\end{aligned}
		$$
		
		% New page for Crowdmark
		\newpage
		
		%Q4
		\item
		\begin{enumerate}
			%Q4(a)
			\item 
			False, disproof by counterexample \newline
			Let $\vec{v} \in \mathbb{R}^n$, pick $\vec{v}=-\vec{u}$ for $\vec{u} \in \mathbb{R}^n$
			$\Rightarrow T(\vec{v})=T(-\vec{u})=\|\vec{u}\|$
			However,
			$$
			-T(\vec{u})=-\|\vec{u}\| \neq\|\vec{u}\|
			$$
			Hence, the statement is false
			%Q4(b)
			\item 
			Direct Proof \newline
			Let $\vec{u}, \vec{v} \in \mathbb{R}^n$
			$$
			\begin{aligned}
			\Rightarrow T(\vec{u}+\vec{v}) & =\vec{x}(\vec{u}+\vec{v}) \\
			& =\vec{u} \vec{x}+\vec{v} \vec{x} \\
			& =T(\vec{u})+T(\vec{v}) \\
			\Rightarrow T(c \vec{v}) & =c \vec{v} \cdot \vec{x} \\
			& =c T(\vec{v})
			\end{aligned}
			$$
			Therefore, $T$ satisfies definition of linear transfornmtion. \newline 
			QED
			%Q4(c)
			\item 
			From Q2.c, we know A is invertible \newline 
			By invertibility criteria, $T_A$ is one to one which implies that whenever
			$$
			\begin{aligned}
			& T_A(\vec{x})=T_A(\vec{y}) \\
			& \Rightarrow \vec{x}=\vec{y}
			\end{aligned}
			$$
			which shows uniqueness. \newline
			By invertibility criteria, $T_A$ is also onto, \newline
			$\Rightarrow$ Range $(T)=\mathbb{F}^n$ and $\ddot{y} \in \mathbb{F}^n$ which shows existence.
			\newline QED
			%Q4(d)
			\item 
			Let $A^{-1} \in M_{u \times n}$ be inverse of $A$
			$$
			\begin{aligned}
			A B & =A^{\top} \\
			A^{-1} A B & =A^{-1} A^{\top} \\
			I_n B & =A^{-1} A^{\top} \\
			B & =A^{-1} A^{\top} \\
			\Rightarrow B^{-1} & =\left(A^{\top}\right)^{-1} A
			\end{aligned}
			$$
			We know $\left(A^{\top}\right)^{-1}$ exists since it is just $\left(A^{-1}\right)^{\top}$
			$$
			\begin{aligned}
			B^{-1} B & =\left(A^{\top}\right)^{-1} A A^{-1} A^{\top} \\
			& =I_n \\
			B B^{-1} & =A^{-1} A^{\top}\left(A^{\top}\right)^{-1} A \\
			& =I_n
			\end{aligned}
			$$
			so $B$ is invertable. \newline
			QED
		\end{enumerate}
		
	\end{enumerate}
\end{document}