\documentclass[11pt]{article}
\usepackage{amsmath,amssymb,amsthm,enumerate,nicefrac,fancyhdr,hyperref,graphicx,adjustbox}
\hypersetup{colorlinks=true,urlcolor=blue,citecolor=blue,linkcolor=blue}
\usepackage[left=2.6cm, right=2.6cm, top=1.5cm, includehead, includefoot]{geometry}
\usepackage[dvipsnames]{xcolor}
\usepackage[d]{esvect}

%% commands
%% useful macros [add to them as needed]
% sets
\newcommand{\C}{{\mathbb{C}}} 
\newcommand{\N}{{\mathbb{N}}}
\newcommand{\Q}{{\mathbb{Q}}}
\newcommand{\R}{{\mathbb{R}}}
\newcommand{\Z}{{\mathbb{Z}}}
\newcommand{\F}{{\mathbb{F}}}

% bases
\newcommand{\mA}{\mathcal{A}}
\newcommand{\mB}{\mathcal{B}}
\newcommand{\mC}{\mathcal{C}}
\newcommand{\mD}{\mathcal{D}}
\newcommand{\mE}{\mathcal{E}}
\newcommand{\mL}{\mathcal{L}}
\newcommand{\mM}{\mathcal{M}}
\newcommand{\mO}{\mathcal{O}}
\newcommand{\mP}{\mathcal{P}}
\newcommand{\mS}{\mathcal{S}}
\newcommand{\mT}{\mathcal{T}}

% linear algebra
\newcommand{\diag}{\operatorname{diag}}
\newcommand{\adj}{\operatorname{adj}}
\newcommand{\rank}{\operatorname{rank}}
\newcommand{\spn}{\operatorname{Span}}
\newcommand{\proj}{\operatorname{proj}}
\newcommand{\prp}{\operatorname{perp}}
\newcommand{\refl}{\operatorname{refl}}
\newcommand{\tr}{\operatorname{tr}}
\newcommand{\nul}{\operatorname{Null}}
\newcommand{\nully}{\operatorname{nullity}}
\newcommand{\range}{\operatorname{Range}}
\renewcommand{\ker}{\operatorname{Ker}}
\newcommand{\col}{\operatorname{Col}}
\newcommand{\row}{\operatorname{Row}}
\newcommand{\cof}{\operatorname{cof}}
\newcommand{\Num}{\operatorname{Num}}
\newcommand{\Id}{\operatorname{Id}}
\newcommand{\ipb}{\langle \thinspace, \rangle}
\newcommand{\ip}[2]{\left\langle #1, #2\right\rangle} % inner products
\newcommand{\M}[2]{M_{#1\times #2}(\F)}
\newcommand{\RREF}{\operatorname{RREF}}
\newcommand{\cv}[1]{\begin{bmatrix} #1 \end{bmatrix}}
\newenvironment{amatrix}[1]{\left[\begin{array}{@{}*{\numexpr#1-1}{c}|c@{}}}{\end{array}\right]}
\newcommand{\am}[2]{\begin{amatrix}{#1} #2 \end{amatrix}}

% vectors
\newcommand{\vzero}{\vv{0}}
\newcommand{\va}{\vv{a}}
\newcommand{\vb}{\vv{b}}
\newcommand{\vc}{\vv{c}}
\newcommand{\vd}{\vv{d}}
\newcommand{\ve}{\vv{e}}
\newcommand{\vf}{\vv{f}}
\newcommand{\vg}{\vv{g}}
\newcommand{\vh}{\vv{h}}
\newcommand{\vl}{\vv{\ell}}
\newcommand{\vm}{\vv{m}}
\newcommand{\vn}{\vv{n}}
\newcommand{\vp}{\vv{p}}
\newcommand{\vq}{\vv{q}}
\newcommand{\vr}{\vv{r}}
\newcommand{\vs}{\vv{s}}
\newcommand{\vt}{\vv{t}}
\newcommand{\vu}{\vv{u}}
\newcommand{\vvv}{{\vv{v}}}
\newcommand{\vw}{\vv{w}}
\newcommand{\vx}{\vv{x}}
\newcommand{\vy}{\vv{y}}
\newcommand{\vz}{\vv{z}}

% display
\newcommand{\ds}{\displaystyle}
\newcommand{\qand}{\quad\text{and}}
\newcommand{\qandq}{\quad\text{and}\quad}
\newcommand{\hint}{\textbf{Hint: }}

% misc
\newcommand{\area}{\operatorname{area}}
\newcommand{\vol}{\operatorname{vol}}
\newcommand{\red}[1]{{\color{red} #1}}
\newcommand{\rc}{\red{\checkmark}}

%% header
\pagestyle{fancy}
\fancyhead[L]{\bf\large MATH136: Linear Algebra 1 \\ Written Assignment 4 Solutions}
\fancyhead[R]{\bf\large Winter 2023 \\}
%\fancyfoot[C]{Page \thepage\ of 2}
\setlength{\headheight}{35pt}

\begin{document}
	\begin{enumerate}[{\bf Q1.}]	
		%Q1
		\item
		\begin{enumerate}
			%Q1(a)
			\item 
			Since $\lambda$ is an eigenvalue of $A$
			$\operatorname{det}(A-\lambda I)=0$ for matrix $I-A$
			$$
			\begin{aligned}
			\operatorname{det}(I-A-(1 \cdot \lambda) I) & =\operatorname{det}(I-A-I+\lambda I) \\
			& =\operatorname{det}(-A+\lambda I) \\
			& =-\operatorname{det}(A-\lambda I) \\
			& =-0 \\
			& =0
			\end{aligned}
			$$
			Hence $1-\lambda$ is an eigenvalue of matrix $I-A$.
			%Q1(b)
			\item 
			Since $\mu$ is an eigenvalue of matrix $I-A$
			$$
			\begin{aligned}
			& \operatorname{det}(I-A-\mu I)=0 . \\
			& \operatorname{det}(-A+(I-\mu I))=0 . \\
			& \operatorname{det}(A-(1-\mu) I)=-0=0 .
			\end{aligned}
			$$
			If $\lambda$ is an eigenvalue of $A$ then $\operatorname{det}(A-\lambda I)=0$
			so there exist $\lambda$ such that
			$$
			\begin{aligned}
			-\lambda & =-(1-\mu) \\
			\mu-1 & =-\lambda \\
			\mu & =1-\lambda
			\end{aligned}
			$$

			%Q1(c)
			\item
			Proof by Contradiction \newline
			Assume $I-A$ is not invertible then $I-A$ must have an eigenvalue where
			$$
			\mu=0
			$$
			According to Q1.b, there exist an eigenvalue $\lambda$ tor $A$ where
			$$
			\begin{aligned}
			& \mu=1-\lambda \\
			& 0=1-\lambda \\
			& \lambda=1
			\end{aligned}
			$$
			Since $|\lambda|<1$ for all $\lambda$ \newline
			$I-A$ is invertible, hence the contradiction and the statement is true. 

		\end{enumerate}
		
		% New page for Crowdmark
		\newpage
		
		%Q2
		\item 
		\begin{enumerate}
			%Q2(a)
			\item 
			When $n=0$.
			$$
			\begin{aligned}
			& a_n=a_0=7 \\
			& a_{n+1}=a_1=26 . \\
			& A^n \cdot \vec{v}=A^0 \cdot \vec{v}=I_2 \cdot \vec{v}=\vec{v}=\left[\begin{array}{l}
			26 \\
			7
			\end{array}\right]=\left[\begin{array}{l}
			a_{n+1} \\
			a_n
			\end{array}\right]
			\end{aligned}
			$$
			Assume that when $n=k \quad k \in \mathbb{R} \wedge k \geqslant 0$.
			$$
			A^k \vec{v}=\left[\begin{array}{l}
			a_{k+1} \\
			a_k
			\end{array}\right]
			$$
			when $n=k+1$
			$$
			\begin{aligned}
			A^{k+1} \vec{v} & =A \cdot A^k \vec{v} \\
			& =\left[\begin{array}{cc}
			7 & -10 \\
			1 & 0
			\end{array}\right]\left[\begin{array}{l}
			a_{k+1} \\
			a_k
			\end{array}\right] \\
			& =\left[\begin{array}{c}
			7 a_{k+1}-10 a_k \\
			a_{k+1}
			\end{array}\right]
			\end{aligned}
			$$
			As provided, $a_n=7 a_{n-1}-10 a_{n-2}$
			$$
			7 a_{k+1}-10 a_k=a_{k+1+1}=a_{k+2} .
			$$
			$$
			\begin{aligned}
			& 7 a_{k+1}-10 a_k=a_{k+1+1}=a_{k+2} . \\
			& \text { so } A^{k+1} \vec{v}=\left[\begin{array}{l}
			a_{k+2} \\
			a_{k+1}
			\end{array}\right]
			\end{aligned}
			$$
			so statement is also true for $n=k+1$
			\newline Hence by the principle of mathematical induction, the statement is true.
			%Q2(b)
			\item 
			For $A=\left[\begin{array}{cc}1 & -10 \\ 1 & 0\end{array}\right]$
			\newline Consider $\operatorname{det}(A-\lambda I)=0$
			$$
			\begin{aligned}
			\operatorname{det}\left(\left[\begin{array}{cc}
			7-\lambda & -10 \\
			1 & -\lambda
			\end{array}\right]\right) & =0 \\
			(7-\lambda)(-\lambda)-1 \times(-10) & =0 \\
			\lambda^2-7 \lambda+10 & =0 \\
			(\lambda-2)(\lambda-5) & =0 \\
			\lambda_1=2 ; \lambda_2 & =5
			\end{aligned}
			$$
			For $\lambda_1=2$. \newline
			Consider $(A-\lambda I) \vec{v}=\overrightarrow{0}$
			$$
			\begin{gathered}
			{\left[\begin{array}{cc}
			5 & -10 \\
			1 & -2
			\end{array}\right] \vec{v}=\overrightarrow{0}} \\
			r_1 \rightarrow r_1-4 r_2 \\
			r_2 \rightarrow r_2-\frac{1}{5} r_1 \\
			{\left[\begin{array}{cc}
			1 & -2 \\
			0 & 0
			\end{array}\right] \vec{v}=\overrightarrow{0}} \\
			x_1-2 x_2=0 \\
			\quad x_2=x_2 \\
			\Rightarrow \vec{v}=x_2\left[\begin{array}{l}
			2 \\
			1
			\end{array}\right]
			\end{gathered}
			$$
			so $\vec{v}_1=\left[\begin{array}{l}2 \\ 1\end{array}\right]$ is an eigenvector of $A$ for $\lambda_1=2$.
			For $\lambda_2=5$. \newline
			Consider $(A-\lambda I) \vec{v}=\overrightarrow{0}$
			$$
			\left[\begin{array}{cc}
			2 & -10 \\
			1 & -5
			\end{array}\right] \vec{v}=\overrightarrow{0} .
			$$
			$$
			\begin{gathered}
			{\left[\begin{array}{cc}
			1 & -5 \\
			0 & 0
			\end{array}\right] \vec{v}=\overrightarrow{0}} \\
			x_1-5 x_2=0 \\
			x_2=x_2 \\
			\vec{v}=x_2\left[\begin{array}{l}
			5 \\
			1
			\end{array}\right]
			\end{gathered}
			$$
			so $\vec{v}_2=\left[\begin{array}{l}5 \\ 1\end{array}\right]$ is an eigenvector of $A$ for $\lambda_2=5$.
			$$
			\begin{aligned}
			& \text { so } p=\left[\begin{array}{ll}
			2 & 5 \\
			1 & 1
			\end{array}\right] \\
			& p^{-1}=\frac{1}{\operatorname{det} p} \cdot\left[\begin{array}{cc}
			1 & -5 \\
			-1 & 2
			\end{array}\right] \\
			& =\frac{1}{2-5} \cdot\left[\begin{array}{cc}
			1 & -5 \\
			-1 & 2
			\end{array}\right] \\
			& =-\frac{1}{3} \cdot\left[\begin{array}{cc}
			1 & -5 \\
			1 & 2
			\end{array}\right] \text {. } \\
			& \text { us } A=P D P^{-1} \\
			& P^{-1} A P=P^{-1} P D P^{-1} P \\
			& D=P^{-1} A P \\
			& =-\frac{1}{3} \cdot\left[\begin{array}{cc}
			1 & -5 \\
			-1 & 2
			\end{array}\right] \cdot\left[\begin{array}{cc}
			7 & -10 \\
			1 & 0
			\end{array}\right] \cdot\left[\begin{array}{ll}
			2 & 5 \\
			1 & 1
			\end{array}\right] \\
			& =-\frac{1}{3}\left[\begin{array}{cc}
			2 & -10 \\
			-5 & 10
			\end{array}\right] \cdot\left[\begin{array}{ll}
			2 & 5 \\
			1 & 1
			\end{array}\right] \\
			& =-\frac{1}{3}\left[\begin{array}{cc}
			-6 & 0 \\
			0 & -15
			\end{array}\right] \\
			& =\left[\begin{array}{ll}
			2 & 0 \\
			0 & 5
			\end{array}\right] \\
			&
			\end{aligned}
			$$

			%Q2(c)
			\item 
			According to Q2.c
			$$
			\begin{aligned}
			& A^n \vec{v}=\left[\begin{array}{l}
			a_{n+1} \\
			a_n
			\end{array}\right] \\
			& \text { so } a_n=\left[\begin{array}{ll}
			0 & 1
			\end{array}\right] \cdot\left[\begin{array}{c}
			a_{w_n}
			\end{array}\right] \\
			& =\left[\begin{array}{ll}
			0 & 1
			\end{array}\right] \cdot A^n \cdot\left[\begin{array}{c}
			26 \\
			7
			\end{array}\right] \\
			& \text { according to (b) } \\
			& a_n=\left[\begin{array}{ll}
			0 & 1
			\end{array}\right]\left(p O P^{-1}\right)^n\left[\begin{array}{c}
			26 \\
			7
			\end{array}\right] \\
			& =\left[\begin{array}{ll}
			0 & 1
			\end{array}\right] \underbrace{\left(P D P^{-1}\right)\left(B D P^{-1}\right) \cdots\left(P D P^{-1}\right)}_{P D P^{-1} \times n}\left[\begin{array}{c}
			26 \\
			7
			\end{array}\right] \\
			& =\left[\begin{array}{ll}
			0 & 1
			\end{array}\right] P \cdot D^n \cdot P^{-1} \cdot\left[\begin{array}{c}
			26 \\
			7
			\end{array}\right] \\
			& =\left[\begin{array}{ll}
			0 & 1
			\end{array}\right] \cdot\left[\begin{array}{ll}
			2 & 5 \\
			1 & 1
			\end{array}\right] \cdot\left[\begin{array}{cc}
			2 & 0 \\
			0 & 5
			\end{array}\right]^n \cdot\left(-\frac{1}{3}\right)\left[\begin{array}{cc}
			1 & -5 \\
			-1 & 2
			\end{array}\right] \cdot\left[\begin{array}{c}
			26 \\
			7
			\end{array}\right] \\
			& =\left[\begin{array}{ll}
			1 & 1
			\end{array}\right] \cdot\left[\begin{array}{cc}
			2^n & 0 \\
			0 & 5^n
			\end{array}\right] \cdot\left(-\frac{1}{3}\right) \cdot\left[\begin{array}{c}
			-9 \\
			-12
			\end{array}\right] \\
			& =\left[\begin{array}{ll}
			2^n & 5^n
			\end{array}\right] \cdot\left[\begin{array}{l}
			3 \\
			4
			\end{array}\right] \\
			& a_n=3 \cdot 2^n+4 \cdot 5^n \\
			&
			\end{aligned}
			$$
			Hence the statement is true.
		\end{enumerate}
		
		% New page for Crowdmark
		\newpage
		
		%Q3 
		\item
		\begin{enumerate}
			%Q3(a)
			\item 
			As $T$ is a linear transformation. If $T(\vec{u}) \neq \vec{o}$ then $\vec{u} \neq \overrightarrow{0}$. Therefore, for all $\vec{w} \in w: \vec{w} \neq \overrightarrow{0}$. $u \neq \overrightarrow{0}$, which means $\overrightarrow{0} \notin S_{\vec{w}}$, so the statement is false.
			%Q3(b)
			\item 
			Since $T$ is a linear transformation.
			$$
			T(\vec{O})=\overrightarrow{0}
			$$
			Since $u$ and $w$ are subspaces of $\mathbb{F}^n$ $\overrightarrow{0} \in U$ and $\overrightarrow{0} \in W$
			$$
			\text { so } \quad \overrightarrow{0} \in S_w
			$$
			For $\vec{x}, \vec{y} \in S w$.
			$$
			\begin{aligned}
			\text { so } & \vec{x} \in U ; T(\vec{x}) \in W \\
			& \vec{y} \in U ; T(\vec{y}) \in W
			\end{aligned}
			$$
			Consider $c \vec{x}+\vec{y}: c \in \mathbb{R}$ \newline
			Since $u$ is a subspace of $\mathbb{F}^n$, $c \vec{x}+\vec{y} \in u$ \newline
			Since $W$ is a subspace of $\mathbb{F}^n$ \newline
			$C\left(\vec{x}^{\prime}\right)+T(\vec{y}) \in W$ \newline
			Since $T$ is a linear transformation.
			$$
			c T(\vec{x})+T(\vec{y})=T((\vec{x}+\vec{y})
			$$
			Hence $T(c \vec{x}+\vec{y}) \in W$, $c \vec{x}+\vec{y} \in S_w$ \newline
			Therefore, the statement is true.
		\end{enumerate}
		
		% New page for Crowdmark
		\newpage
		
		%Q4
		\item
		\begin{enumerate}
			%Q4(a)
			\item 
			Since $\left\{\overrightarrow{b_1}, \cdots, \overrightarrow{b_k}\right\}$ is linearly independent then $c_1 \vec{b}_1+\cdots+c_k \vec{b}_k=\overrightarrow{0}$ only has trivial solution. Since $\vec{x}_i$ is a solution to $A \vec{x}=\vec{b}_i$ for $i=1 \ldots, k$ $A \overrightarrow{x_i}=\overrightarrow{b_1}$ \newline
			$c_1 A \vec{x}_1+\cdots+c_k A \vec{x}_k=\overrightarrow{0}$ only has trivial solution. \newline
			$A\left(c_1 \vec{x}_1+\cdots+c_k \vec{x}_k\right)=\overrightarrow{0}$ only has trivial solution. \newline
			Since $\left\{\overrightarrow{b_1}: \cdots, \overrightarrow{b_{11}}\right\}$ is linearly independent,
			so there must exist $\overrightarrow{b_i} \neq 0$. \newline
			Also we have $A \vec{x}_i=\vec{b}$, so $\operatorname{rank}(A) \neq 0$. \newline
			Modifying $A\left(c_1 \overrightarrow{x_1}+\cdots+c_k \vec{x}_k\right)=\overrightarrow{0}$ we can have
			$c_1 \vec{x}+\cdots+c_k \overrightarrow{x_k}=\overrightarrow{0}$ has only trivial solution \newline
			hence $\left\{\vec{x}_1, \cdots, \vec{x}_k\right\}$ is linearly independent.
			%Q4(b)
			\item 
			consider $c_1 \vec{b}_1+\cdots+c_k \vec{b}_k=\overrightarrow{0}$.
			$$
			\begin{aligned}
			c_1 A \overrightarrow{x_1}+\cdots+c_k A \overrightarrow{x_k} & =\overrightarrow{0} . \\
			A\left(c_1 \overrightarrow{x_1}+\cdots+c_k \overrightarrow{x_k}\right) & =\overrightarrow{0} \\
			c_1 \overrightarrow{x_1}+\cdots+c_k \overrightarrow{x_k} & \in \operatorname{Null}(A)
			\end{aligned}
			$$
			Since $\operatorname{rank}(A)=n$, Null $(A)=\{\overrightarrow{0}\}$ \newline
			so $A\left(c_1 \vec{x}_1+\cdots+c_k \overrightarrow{x_k}\right)=\overrightarrow{0}$, if and only if $c_1 \overrightarrow{x_1}+\cdots+c_{k} \overrightarrow{x_{k}}=\overrightarrow{0}$.
			as $\left\{\vec{x}_1, \cdots, \vec{x}_k\right\}$ is linearly independent. \newline
			So $A\left(c_1 \overrightarrow{x_1}+\cdots+c_k \vec{x}_k\right)=0$ has only trivial solution, therefore, $\left\{\overrightarrow{b_1}, \cdots, \overrightarrow{b_k}\right\}$ is linearly independent.
		\end{enumerate}
	\end{enumerate}
\end{document}